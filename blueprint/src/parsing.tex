\subsection{Parsing}
Results related to construction of the PDA
\begin{definition}[Parser]
    \label{def:Parser}
    \lean{PDA.fromCFG}
    \uses{def:PDA}
    Given an input context free grammar, build a pushdown automata that recognizes the CFG. Delaying details of the algorithm until Lean formalization time, but I'm sure there's a well known construction.
\end{definition}

\begin{theorem}[Parser Equivalent to CFG] 
    \label{thm:ParserEquivCFG}
    \lean{PDA.fromCFGCorrect}
    \uses{def:Parser,def:CFG}
    Let $\pushdown$ be the PDA as constructed above. A terminal sequence is accepted by $\pushdown$ if and only if it is a member of the input CFG.
\end{theorem}

\begin{theorem}[Parser Lexer Equivalent to CFG Sentences] 
    \label{thm:ParserLexerEquivCFGSentences}
    \uses{def:Parser,def:CFGSentences,thm:ParserEquivCFG}
    Let $\pushdown \circ \transducer_{\automaton}$ denote the composition between the parser and lexing transducer.
    An input string $w \in \alphabet^{*}$ is accepted by $\pushdown \circ \transducer_{\automaton}$ if and only if it is a sentence in the original CFG (with respect to $\lexer$).
\end{theorem}

\begin{theorem}[Parser Lexer Detokenizer Equivalent to CFG Sentences] 
    \label{thm:ParserLexerDetokenizerEquivCFGSentences}
    \uses{def:Parser,thm:ParserLexerEquivCFGSentences}
    Let $\pushdown \circ \transducer_{\automaton \circ \vocab}$ denote the composition between the parser and lexing transducer and detokenizer.
    An input string $w \in \vocab^{*}$ is accepted by $\pushdown \circ \transducer_{\automaton \circ \vocab}$ if and only if $F(w)$ (the string over $\alphabet^{*}$) is a sentence in the original CFG (with respect to $\lexer$).
\end{theorem}

\begin{definition}[PDA Pruned]
    \label{def:PDAPruned}
    \lean{PDA.pruned}
    \leanok
    \uses{def:PDA}
    A PDA is pruned if for every state $q$ and stack configuration $\gamma$ reachable from $q_0$, there exists a sequence of terminals such that the PDA starting in state $q$ with stack configuarion $\gamma$ reaches an accepting state.
\end{definition}

\begin{theorem}{Intermediate Pruned PDA Equiv Prefix CFG}
    \label{thm:IntermediatePrunedPDAPrefixCFG}
    \lean{PDA.fromCFGPruned}
    If the language of a parser is equal to the language of a context free grammar and the pda is pruned, then the intermediate language (all strings that the parser can process from the initial state) is equal to the prefix language of the CFG.
\end{theorem}

\begin{lemma}[PDA Stack Invariance]
    \label{lem:PDAStackInvariance}
    \lean{PDA.stackInvariance}
    \leanok
    If a PDA $\pushdown$ accepts an input sequence $w$ in state $q$ with stack configuration $\gamma$, then $w$ is also accepted in the same state $q$ when the stack configuration is $\gamma' \cdot \gamma$ for some $\gamma'$ (i.e., when $\gamma$ appears at the top of the stack with additional symbols beneath it).
\end{lemma}
\begin{proof}
    \leanok
\end{proof}

\begin{lemma}[PDA Over Approximation]
    \label{lem:PDAOverApproximation}
    \lean{PDA.overApproximation}
    \leanok
    If a PDA $\pushdown$ with all stack operations removed does not accept an input sequence $w$ under initial state $q$, then irrespective of the initial stack configuration $\gamma$, the original PDA also does not accept the input sequence $w$ under initial state $q$ and stack configuration $\gamma$. 
\end{lemma}
\begin{proof}
    \leanok
\end{proof}
