
\subsection{Grammar Constrained Decoding}

\begin{definition}[Find Always Accepted Tokens]
    \label{def:FindAlwaysAccepted}
    Algorithm that finds the always accepted tokens
\end{definition}


\begin{lemma}[Find Always Accepted Tokens Correct]
    \label{lem:FindAlwaysAcceptedCorrect}
    \uses{def:FindAlwaysAccepted}
    The algorithm for finding the always accepted tokens is exactly the set of always accepted tokens.
\end{lemma}
\begin{proof}
    \uses{lem:PDAStackInvariance} 
\end{proof}

\begin{definition}[Find Context Dependent Terminal Sequences]
    \label{def:FindContextDependentTerminalSequences}
    Algorithm that finds the context dependent terminal sequences 
\end{definition}

\begin{lemma}[Find Context Dependent Correct]
    \label{lem:FindContextDependentCorrect}
    \uses{def:FindContextDependentTerminalSequences}
    The algorithm for finding the context dependent terminal sequences is exactly the set of context dependent terminal sequences (in particular, all terminal sequences that may possibly be accepted are included in this set).
\end{lemma}

\begin{definition}[ComputeValidTokenMask]
    \label{alg:ComputeTokenMask}
    \[
        \begin{array}{l}
            \textbf{Input:} \text{PDA } \mathcal{P}, \text{ FSA } \mathcal{A}, \text{ Inverse token spanner table } T_{\text{inv}} \\
            \textbf{Input:} \text{Always-accepted token table } A, \\
            \textbf{Input:} \text{Context-dependent sequence table } D, \\
            \textbf{Input:} \text{Lexer state } q^{\mathcal{A}}, \text{ Parser state } q^{\mathcal{P}}, \text{ Stack configuration } \gamma \\
            \textbf{Output:} \text{Set of allowed tokens } \mathcal{V}_{\text{allowed}} \\
            \\
            1: \mathcal{V}_{\text{allowed}} := A(q^{\mathcal{A}}, q^{\mathcal{P}}) \\
            2: \text{for each } \alpha \in D(q^{\mathcal{A}}, q^{\mathcal{P}}) \text{ do} \\
            3: \quad \text{if } \alpha \text{ is accepted by } \mathcal{P} \text{ in state } q^{\mathcal{P}} \text{ with stack } \gamma \text{ then} \\
            4: \quad\quad \mathcal{V}_{\text{allowed}} := \mathcal{V}_{\text{allowed}} \cup T_{\text{inv}}(q^{\mathcal{A}}, \alpha) \\
            5: \quad \text{end if} \\
            6: \text{end for} \\
            7: \text{return } \mathcal{V}_{\text{allowed}}
        \end{array}
    \]
\end{definition}

\begin{lemma}[ComputeTokenMaskCorrect]
    \label{lem:ComputeTokenMaskCorrect}
    A token is a member of the token mask produced ComputeValidTokenMasks if and only if the composite $\pushdown \circ \transducer_{\automaton \circ \vocab}$ accepts the token in the current lexer and parser state
\end{lemma}

\begin{lemma}
    \label{lem:completeness-lemma}
    Given $w \in \vocab^\ast$ such that $q \xrightarrow{w: T_1 \ldots T_k}^\ast q'$ via the lexing transducer and $T_1 \ldots T_k \in \lang(\grammar) \setminus \langpre(\grammar)$,
    if $w \in \langpre^\lexer(\grammar)$ then there exists $T \in \producible(q')$ such that $T_1 \ldots T_k T \in \langpre(\grammar)$.
\end{lemma}

\begin{proof}
    Suppose $w \in \langpre^\lexer(\grammar)$. Then there exists $v$ such that $wv \in \lang^\lexer(\grammar)$.
    This implies $\lexer(wv) = (T_1 \ldots T_k T_{k+1} \ldots T_m \$, \epsilon)$ and
    $T_1 \ldots T_k T_{k+1} \ldots T_m \$ \in \lang(\grammar)$.
    Since $T_1 \ldots T_k \notin \lang(\grammar)$, it follows that $T_{k+1} \ldots T_m \neq \epsilon$, and in particular $T_{k+1} \in \producible(q')$.
    Therefore, $T_1 \ldots T_k T_{k+1} \in \langpre(\grammar)$.
\end{proof}

\begin{theorem}[Token Mask Completeness]
    \label{thm:TokenMaskCompleteness}
    Let $\pushdown = (\terms, \Pi, Q^{\pushdown}, q_0^{\pushdown}, Z_0, \delta^{\pushdown}, F^{\pushdown})$ be a pushdown automaton such that $\lang(\pushdown) = \lang(\grammar)$ and $\pushdown$ is pruned.
    Given a string $w \in \langpre^\lexer(\grammar)$ and a token $v \in \vocab$, the token $v$ is not masked by \autoref{alg:ComputeTokenMask}  (i.e., $v \in \vocab_{\textit{allowed}}$) if the concatenation $wv$ belongs to $\langpre^\lexer(\grammar)$.
\end{theorem}

\begin{proof}
    \uses{lem:completeness-lemma, thm:LexingTransducerMemoryless, thm:IntermediatePrunedPDAPrefixCFG, lem:ComputeTokenMaskCorrect}
    By \autoref{thm:LexingTransducerMemoryless}, the lexing transducer emitted $T_1 \ldots T_k$ and reached a state $q'$ such that $q_0 \xrightarrow{w_r: \epsilon}^\ast q'$.
    If $wv \in \langpre^\lexer(\grammar)$, then $q_0 \xrightarrow{wv: T_1 \ldots T_k T_{k+1} \ldots T_m} q''$ for some state $q''$, which implies $q' \xrightarrow{T_{k+1} \ldots T_m} q''$.
    By Lemma~\ref{lem:completeness-lemma}, there exists $T \in \producible(q'')$ such that $T_1 \ldots T_k T_{k+1} \ldots T_m T \in \langpre(\grammar)$.
    Therefore, $v \in \inversetable(q', T_{k+1} \ldots T_m T)$, and hence $v \in \vocab_{\textit{allowed}}$.
\end{proof}

\begin{lemma}[Token Mask Soundness]
    \label{lem:soundness-lemma}
    Given $w \in \vocab^\ast$ such that $q \xrightarrow{w: T_1 \ldots T_k}^\ast q'$ via the lexing transducer and $T_1 \ldots T_k \in \lang(\grammar) \setminus \langpre(\grammar)$,
    if there exists $T \in \producible(q')$ such that $T_1 \ldots T_k T \in \langpre(\grammar)$ then $w \in \langpre^\lexer(\grammar)$. 
\end{lemma}
\begin{proof}
    This uses the assumption the all terminal sequences are realizable.
\end{proof}

\begin{theorem}[Token Mask Soundness]
    \label{thm:TokenMaskSoundness}
    Let $\pushdown = (\terms, \Pi, Q^{\pushdown}, q_0^{\pushdown}, Z_0, \delta^{\pushdown}, F^{\pushdown})$ be a pushdown automaton such that $\lang(\pushdown) = \lang(\grammar)$.
    Given a string $w \in \langpre^\lexer(\grammar)$ and a token $v \in \vocab$, the token $v$ is not masked by \autoref{alg:ComputeTokenMask}  (i.e., $v \in \vocab_{\textit{allowed}}$) only if the concatenation $wv$ belongs to $\langpre^\lexer(\grammar)$.
\end{theorem}

\begin{proof}
    \uses{lem:ComputeTokenMaskCorrect, lem:soundness-lemma}
\end{proof}

\begin{definition}[GCDChecker]
    \label{def:GCDChecker}
    \uses{alg:ComputeTokenMask, def:Parser, def:LexingTransducer, def:FindAlwaysAccepted, def:FindContextDependentTerminalSequences}
    We are given the lexer specification and the context free grammar as inputs.
    The actual GCD checker is constructed as follows. 
    First, build the composite transducer $\transducer_{\alphabet \circ \vocab}$
    Next, build the parser $\pushdown$ associated with the CFG. Then compute the set of always accepted and context-dependent tokens for each lexer and parser state.

    The actual GCD checker then returns the set of allowed tokens by calling \ref{alg:ComputeTokenMask}
\end{definition}

\begin{theorem}[Prefix Language of Grammar subset Intermediate Language GCDChecker]
    \label{thm:PrefixSubsetIntermediate} 
    The prefixes of sentences of the CFG is a subset of the intermediate language of the GCD checker, modulo conversion between $\alphabet$ and $\vocab$.
\end{theorem}
\begin{proof}
    \uses{thm:TokenMaskCompleteness}
    Follows from \ref{thm:TokenMaskCompleteness}
\end{proof}


\begin{theorem}[Intermediate Language GCDChecker subset Prefix Language of Grammar]
    \label{thm:IntermediateSubsetPrefix} 
    The intermediate language of the GCD checker is a subset of the prefixes of sentences of the CFG, modulo conversion between $\alphabet$ and $\vocab$.
\end{theorem}
\begin{proof}
    \uses{thm:TokenMaskSoundness}
    Follows from \ref{thm:TokenMaskSoundness}
\end{proof}

\begin{theorem}[Language of Grammar subset of GCD Checker]
    \label{thm:GrammarSubsetCheckerLang}
    The sentences of the CFG is a subset of the language of the GCD checker, modulo conversion between $\alphabet$ and $\vocab$.
\end{theorem}
\begin{proof}
    \uses{thm:PrefixSubsetIntermediate}
    Follows mainly by \ref{thm:PrefixSubsetIntermediate}.
\end{proof}

\begin{theorem}[Language of GCD Checker subset of Grammar]
    \label{thm:CheckerLangSubsetGrammar}
    The language of the GCD checker is a subset of the sentences of the CFG, modulo conversion between $\alphabet$ and $\vocab$.
\end{theorem}
\begin{proof}
    \uses{def:Parser}
    This direction should be trivial, since end of string is allowed if and only if parser can go to the accepting state. 
\end{proof}

Here are the final theorems that we want to ultimately prove. \ref{thm:LexSplit} may also be included as a final theorem.
\begin{theorem}[GCDChecker Complete]
    \label{thm:GCDCheckerComplete}
    \uses{def:CheckerComplete,def:CFGSentences,thm:GrammarSubsetCheckerLang,thm:CheckerLangSubsetGrammar, thm:PrefixSubsetIntermediate, thm:IntermediateSubsetPrefix}
    The GCD Checker is complete with respect to the language (sentences) of the CFG.
\end{theorem}
\begin{proof}
    Follows from above lemmas.
\end{proof}

\begin{theorem}[GCDChecker Sound]
    \label{thm:GCDCheckerSound}
    \uses{def:CheckerSound}
    The GCD checker is in fact sound.
\end{theorem}
\begin{proof}
    The first property follows from completeness. The second one should be provable based on the fact that the lexing transducer itself is path independent? Shouldn't be too difficult I don't think
\end{proof}

