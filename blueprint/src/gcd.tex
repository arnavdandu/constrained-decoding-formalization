\section{Grammar Constrained Decoding}

\subsection{Background}
\begin{definition}[Finite State Automata]
    \label{def:FSA}
    See Mathlib
\end{definition}

\begin{definition}[Finite State Transducer]
    \label{def:FST}
    FST is an FSA except that transitions can have labels onto an output alphabet.
\end{definition}

\begin{definition}[Context Free Grammar]
    \label{def:CFG}
    See Mathlib. 
\end{definition}

\begin{definition}[Lexer]
    \label{def:Lexer}
    A lexer is a function from $\alphabet^{*}$ to $\terms^{*} \times \alphabet^{*}$, denoting the lexed and unlexed portions of the string

\begin{definition}[CFG Sentences]
    \label{def:CFGSentences}
    \uses{def:CFG,def:Lexer}
    The sentences of a CFG with respect to some function $\lexer$ is the set of strings $w$ such that $\lexer(w)$ is a member of the context free grammar.
\end{definition}

\begin{definition}[Pushdown Automata]
    \label{def:PDA}
    A \emph{pushdown automaton} is a tuple $P = (\Sigma, \Pi, Q, q_0, Z_0, \delta, F)$ where:
    \begin{itemize}
        \item $\Sigma$ is the input alphabet,
        \item $\Pi$ is the stack alphabet,
        \item $Q$ is the set of states,
        \item $q_0 \in Q$ is the initial state,
        \item $Z_0 \in \Pi$ is the initial stack symbol,
        \item $F \subseteq Q$ is the set of accepting states,
        \item $\delta \subseteq Q \times (\Sigma \cup \{\epsilon\}) \times \Pi^* \times Q \times \Pi^*$ is the transition relation.
    \end{itemize}

    Each transition $(q, c, \alpha, q', \beta)$ specifies that, in state $q$, upon reading the input symbol $c$ and matching the top stack symbols to $\alpha$, the PDA transitions to state $q'$ and replaces $\alpha$ with the sequence $\beta$ on the stack.
\end{definition}

\subsection{Lexing}
Results related to the lexing and detokenizing.

\begin{definition}[Lexer Specification]
	\label{def:LexerSpecification}
    A \emph{lexer specification} is given as a finite set of pairs, each consisting of an automaton $\automaton^i = (Q^i, \alphabet^i, q_{0}^{i}, \delta^i, F^i)$ and a terminal symbol $T^i \in \terms$.
\end{definition}
If we want to make a connection between regular expressions and finite state automata, here would be the place. In particular, we would change the lexer specification from a set of FSA, to a set of regular expressions (and then must prove the conversion process to automata preserves the language).

\begin{definition}[Partial Lexer]
    \label{def:PartialLexer}
    \uses{def:LexerSpecification}
    \begin{itemize}
    \item $\lexer(\epsilon) = (\epsilon, \epsilon)$
    \item Given $\lexer(w) = (T_1 \ldots T_k, w_r)$, the value of $\lexer(wc)$ is
        \begin{itemize}
        \item $(T_1 \ldots T_k T^j \$, \epsilon)$ if $c = \eos$ and $w_r \in \lang(\automaton^j)$ for some $j$
        \item $(T_1 \ldots T_k, w_r c)$ if $c \neq \eos$ and $w_r c \in \langpre(\automaton^i)$ for some $i$
        \item $(T_1 \ldots T_k T^j, c)$ if $c \neq \eos$ and $w_r \in \lang(\automaton^j)$ but $w_r c \notin \langpre(\automaton^i)$ for all $i$
        \item $\bot$ otherwise % (i.e., $w_r \notin \lang(\automaton^i)$ and $w_r c \notin \langpre(\automaton^i)$ for all $i$)
        \end{itemize}
    \end{itemize}
\end{definition}

\begin{theorem}[\lexer returns iff input is Language of Lexer Specification]
    \label{thm:LexSplit}
    \uses{def:PartialLexer,def:LexerSpecification}
    Let $\lang(A_i)$ be the languages of each input automata.
    If $\lexer$ fails to return on some input string $w$, then among all possible partitions of $w$ into a set of strings $w_1 \ldots w_k$, there exists an $i$ such that $w_i \notin \lang(A_j)$ for all j. Conversely, if $\texttt{Lex} = (T_1, \ldots T_k, \epsilon)$, then $w$ may be partitioned into $w_1 \ldots w_k$ such that $w$ is the concatenation of $w_1 \ldots w_k$ and each $w_i \in \lang(A_j)$ for some $j$.
\end{theorem}
We can possibly make the second direction stronger and mention how the partition it produces is related to maximal munching / 1-lookahead.

\begin{definition}[DetokenizingFST]
    \label{def:DetokenizingFST}
    \uses{def:Finite-State Transducer, def:Detokenizer}
    \begin{array}{l}
        \textbf{Input:} \text{Vocabulary } \mathcal{V} \subseteq \Sigma^+ \\
        \textbf{Output:} \text{FST } \mathcal{T}_\mathcal{V} = (\mathcal{V}, \Sigma, Q, q_0, \delta, F) \\
        Q := \{q_\epsilon\},\quad F := \{q_\epsilon\},\quad q_0 := q_\epsilon,\quad \delta := \emptyset \\
        \text{for } c_1 \ldots c_k \in \mathcal{V} \text{ do} \\
        \quad q_{\text{prev}} := q_\epsilon \\
        \quad \text{for } i = 1 \text{ to } k-1 \text{ do} \\
        \quad\quad Q := Q \cup \{q_{c_1 \ldots c_i}\} \\
        \quad\quad \delta := \delta \cup \{ q_{\text{prev}} \xrightarrow{c_i:\epsilon} q_{c_1 \ldots c_i} \} \\
        \quad\quad q_{\text{prev}} := q_{c_1 \ldots c_i} \\
        \quad \delta := \delta \cup \{ q_{\text{prev}} \xrightarrow{c_k:c_1\ldots c_k} q_\epsilon \} \\
        \textbf{return } \mathcal{T}_\mathcal{V} = (\mathcal{V}, \Sigma, Q, q_0, \delta, F)
    \end{array}
\end{definition}

\begin{definition}[Lexing Transducer]
    \label{def:LexingTransducer}
    \uses{def:Finite-State Automata, def:Lexing Transducer}
    \begin{array}{l}
        \textbf{Input:} \text{FSA } \mathcal{A} = (\Sigma, Q, q_0, \delta, F), \text{ Output alphabet } \Gamma \\
        \textbf{Output:} \text{FST } \mathcal{T}_A = (\Sigma, \Gamma, Q, q_0, \delta_{\text{FST}}, F_{\text{FST}}) \\
        \delta_{\text{FST}} := \{ q \xrightarrow{c:\epsilon} q' \mid q \xrightarrow{c} q' \in \delta \} \\
        F_{\text{FST}} := \{q_0\} \\
        \text{for state } q \text{ that recognizes language token } T \in \Gamma \text{ do} \\
        \quad \text{for } (c, q') \text{ s.t. } \exists q''. q \xrightarrow{c} q'' \notin \delta \text{ and } q_0 \xrightarrow{c} q' \in \delta \text{ do} \\
        \quad\quad \delta_{\text{FST}} := \delta_{\text{FST}} \cup \{ q \xrightarrow{c:T} q' \} \\
        \quad \delta_{\text{FST}} := \delta_{\text{FST}} \cup \{ q \xrightarrow{\text{EOS}:T\Phi} q_0 \} \\
        \textbf{return } \mathcal{T}_A = (\Sigma, \Gamma, Q, q_0, \delta_{\text{FST}}, F_{\text{FST}})
    \end{array}
\end{definition}

\begin{theorem}[Lexing Transducer Equivalent to \lexer]
    \label{thm:LexingTransducerLexEquiv}
    \uses{def:PartialLexer, def:LexingTransducer}
    \begin{enumerate}
        \item If $\lexer(w)$ fails to return for some input string $w$, then the transducer does not accept $w$
        \item If $\lexer(w) = (T_1, \ldots T_k, \epsilon)$, then the transducer produces $T_1 \ldots T_k$ upon reading $w$.
    \end{enumerate}
\end{theorem}
\begin{proof}
    (incomplete)
    Let \(\mathcal{T}_A = (Q, \Sigma, \Gamma, q_0, \delta, F)\) be a lexing transducer for the lexer specification \(\{(\mathcal{A}^i, T^i)\}_i\). Then 
\[
    q_0 \xrightarrow{w : T_1 \ldots T_k}{}^* q' \in \delta^* \quad \text{if and only if} \quad \lexer(w) = (T_1 \ldots T_k, w_r)
    \]
    for some \(w_r \in \Sigma^*\) and \(q' \in Q\) such that \(q_0 \xrightarrow{w_r : \epsilon}{}^* q'\).
\end{proof}

\begin{theorem}[Composite Transducer Equivalent to \lexer]
    \label{thm:CompositeLexTransducerLexEquiv}
    Let $\transducer_{\automata \circ \vocab}$ denote the (determinized) composition of the lexer and transducer. Let $F : \vocab^{*} \to \alphabet^{*}$ denote the map that transforms strings over $\vocab$ to strings over $\alphabet$. Let $v \in \vocab^{*}$.

    \begin{enumerate}
        \item If $\lexer(F(v))$ fails to return, then $\transducer_{\automata \circ \vocab}$ does not accept $v$
        \item If $\lexer(F(v)) = (T_1, \ldots T_k, \epsilon)$, then the transducer produces $T_1 \ldots T_k$ upon reading $v$.
    \end{enumerate}
\end{theorem}
\begin{proof}
    This should largely follow from the correctness of composition (and will likely be a lemma)
\end{proof}

\subsection{Parsing}
Results related about steps downstream of lexing.
\begin{definition}[Parser]
    \label{def:Parser}
    Given an input context free grammar, build a pushdown automata that recognizes the CFG. Delaying details of the algorithm until Lean formalization time, but I'm sure there's a well known construction.
\end{definition}

\begin{theorem}[PDA Equivalent to CFG] 
    \label{thm:PDAEquivCFG}
    \uses{def:Parser}
    Let $\pushdown$ be the PDA as constructed above. A terminal sequence is accepted by $\pushdown$ if and only if it is a member of the input CFG.
\end{theorem}

\begin{theorem}[PDA Lexer Equivalent to CFG Sentences] 
    \label{thm:PDALexerEquivCFGLang}
    \uses{def:Parser,def:CFGSentences}
    Let $\pushdown \circ \transducer_{\automata}$ denote the composition between the parser and lexing transducer.
    An input string $w \in \alphabet^{*}$ is accepted by $\pushdown \circ \transducer_{\automata}$ if and only if it is a sentence in the original CFG (with respect to $\lexer$).
\end{theorem}

\begin{theorem}[PDA Lexer Detokenizer Equivalent to CFG Sentences] 
    \label{thm:PDALexerDetokenizerEquivCFGLang}
    \uses{def:Parser,}
    Let $\pushdown \circ \transducer_{\automata \circ \vocab}$ denote the composition between the parser and lexing transducer and detokenizer.
    An input string $w \in \vocab^{*}$ is accepted by $\pushdown \circ \transducer_{\automata \circ \vocab}$ if and only if $F(w)$ (the string over $\alphabet^{*}$) is a sentence in the original CFG (with respect to $\lexer$).
\end{theorem}

\subsection{Grammar Constrained Decoding}
\begin{definition}[GCDChecker]
    \label{def:GCDChecker}
    \uses{def:}
    The actual GCD checker is constructed as follows. We are given the lexer specification and the context free grammar.
    First, build the composite transducer $\transducer_{\alphabet \circ \vocab}$
    Next, build the parser $\pushdown$ associated with the CFG.
    The actual gcd checker is given by first producing the lexing transducer
    todo
\end{definition}

\begin{theorem}[Language of Grammar subset of GCD Checker]
    \label{thm:GrammarSubsetCheckerLang}
    The sentences of the CFG is a subset of the language of the GCD checker, modulo conversion between $\alphabet$ and $\vocab$.
\end{theorem}

\begin{theorem}[Language of GCD Checker subset of Grammar]
    \label{thm:CheckerLangSubsetGrammar}
    The language of the GCD checker is a subset of the sentences of the CFG, modulo conversion between $\alphabet$ and $\vocab$.
\end{theorem}
\begin{proof}
    This direction is trivial 
\end{proof}

\begin{theorem}[Prefix Language of Grammar subset Intermediate Language GCDChecker]
    \label{thm:PrefixSubsetIntermediate} 
    The prefixes of sentences of the CFG is a subset of the intermediate language of the GCD checker, modulo conversion between $\alphabet$ and $\vocab$.
\end{theorem}

\begin{theorem}[Intermediate Language GCDChecker subset Prefix Language of Grammar]
    \label{thm:IntermediateSubsetPrefix} 
    The intermediate language of the GCD checker is a subset of the prefixes of sentences of the CFG, modulo conversion between $\alphabet$ and $\vocab$.
\end{theorem}

Here are the final theorems that we want to ultimately prove. \ref{thm:LexSplit} may also be included as a final theorem.
\begin{theorem}[GCDChecker Complete]
    \label{thm:GCDCheckerComplete}
    \uses{def:CheckerSound,def:CFGLang}
    The GCD Checker is complete with respect to the language (sentences) of the CFG.
\end{theorem}
\begin{proof}
    Follows from above lemmas.
\end{proof}

\begin{theorem}[GCDChecker Sound]
    \label{thm:GCDCheckerSound}
    \uses{def:CheckerSound}
    The GCD checker is in fact sound.
\end{theorem}
\begin{proof}
    The first property follows from completeness. The second one should be provable based on the fact that the lexing transducer itself is path independent? We'll have to think about it more.
\end{proof}

