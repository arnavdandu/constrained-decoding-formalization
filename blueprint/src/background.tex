\subsection{Background}
We introduce some well known concepts and proofs.
\begin{definition}[Finite State Automata]
    \label{def:FSA}
    \leanok
    See Mathlib
\end{definition}

\begin{definition}[Regular Expressions]
    \label{def:RE}
    \leanok
    See Mathlib
\end{definition}

\begin{definition}[Finite State Transducer]
    \label{def:FST}
    \lean{FST}
    \leanok
    FST is an FSA except that transitions can have labels onto an output alphabet.
\end{definition}

\begin{definition}[Context Free Grammar]
    \label{def:CFG}
    \leanok
    See Mathlib. 
\end{definition}

\begin{definition}[Lexer]
    \label{def:Lexer}
    \lean{Lexer}
    \leanok
    A lexer is a function from $\alphabet^{*}$ to $\terms^{*} \times \alphabet^{*}$, denoting the lexed and unlexed portions of the string
\end{definition}

\begin{definition}[CFG Sentences]
    \label{def:CFGSentences}
    \lean{cfgSentences}
    \leanok
    \uses{def:CFG,def:Lexer}
    The sentences of a CFG, denoted $\lang^{\lexer}(\grammar)$, with respect to some function $\lexer$ is the set of strings $w$ such that $\lexer(w)$ is a member of the context free grammar.
\end{definition}


\begin{definition}[CFG Prefix Language]
    \label{def:CFGPrefixLanguage}
    \lean{cfgSentencesPre}
    \leanok
    \uses{def:CFG,def:Lexer,def:CFGSentences}
    The (lexer) prefix language of a grammar $\grammar$, denoted $\langpre^\lexer(\grammar)$, is the set of all prefixes of sentences of the grammar.
\end{definition}

\begin{definition}[Pushdown Automata]
    \label{def:PDA}
    \lean{PDA}
    \leanok

    A \emph{pushdown automaton} is a tuple $P = (\Sigma, \Pi, Q, q_0, Z_0, \delta, F)$ where:
    \begin{itemize}
        \item $\Sigma$ is the input alphabet,
        \item $\Pi$ is the stack alphabet,
        \item $Q$ is the set of states,
        \item $q_0 \in Q$ is the initial state,
        \item $Z_0 \in \Pi$ is the initial stack symbol,
        \item $F \subseteq Q$ is the set of accepting states,
        \item $\delta \subseteq Q \times (\Sigma \cup \{\epsilon\}) \times \Pi^* \times Q \times \Pi^*$ is the transition relation.
    \end{itemize}

    Each transition $(q, c, \alpha, q', \beta)$ specifies that, in state $q$, upon reading the input symbol $c$ and matching the top stack symbols to $\alpha$, the PDA transitions to state $q'$ and replaces $\alpha$ with the sequence $\beta$ on the stack.

\end{definition}
