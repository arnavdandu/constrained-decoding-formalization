\subsection{Constrained Decoding}
General propositions about constrained decoding

\begin{definition}[Vocabulary]
    \label{def:Vocabulary}
    $\vocab$ is said to be a vocabularly over a base alphabet $\alphabet$ if $\alphabet \subseteq \vocab \subseteq\alphabet^{*}$.
\end{definition}

\begin{definition}[Extended Vocabulary]
    \label{def:XVocabulary}
    \uses{def:Vocabulary}
    The extended vocabularly $\xvocab$ is the vocabularly $\vocab$ adjoined with the end of string token, i.e. $\vocab \ \cup \ \{\$\}$
\end{definition}

\begin{definition}[Language Model]
    \label{def:LanguageModel}
    \uses{def:XVocabulary}
    
    A \emph{language model} is given by some function $L: \alphabet^{*} \to \xvocab$.
\end{definition}

\begin{definition}[Checker]
    \label{def:Checker}
    \uses{def:XVocabulary}
    A \emph{checker} is a function $C: \alphabet^* \to \mathbb{P}(\xvocab)$ which specifies the set of tokens that may be produced given the currently produced string
\end{definition}

\begin{definition}[Constrained Decoding]
    \label{def:ConstrainedDecoding}
    \uses{def:LanguageModel,def:Checker,def:Vocabulary}
    \[
    \begin{array}{l}
    \textbf{Algorithm: ConstrainedDecoding} \\
    \textbf{Input: } \text{Model } M, \text{Checker } C, \text{Tokenized prompt } x \\
    \mathcal{V} := M.\texttt{vocabulary} \\
    \textbf{repeat} \\
    \quad m := C(x) \\
    \quad \textit{logits} := M(x) \\
    \quad t_{\text{next}} := \texttt{sample}(\texttt{applyMask}(m, \textit{logits})) \\
    \quad x := x.\texttt{append}(t_{\text{next}}) \\
    \textbf{until } t_{\text{next}} \ne \texttt{EOS} \\
    \textbf{return } x \\
    \end{array}
    \]
\end{definition}



\begin{definition}[Checker Intermediate Language]
    \label{def:CheckerIntermediateLanguage}
    \uses{def:Checker,def:Vocabulary}
    The checker intermediate language for some checker $C$ is defined recursively:
    \begin{itemize}
        \item $\epsilon \in \langint$
        \item If $w \in \langint$ and $v \in \vocab, v \in C(w)$, then $wv \in \langint$
    \end{itemize}
\end{definition}

Theoretically, another way of formalizing this would be to consider the IntermediateLanguage \emph{parameterized} by the initial prompt (versus right now where we only start from $\epsilon$). In the general constrained decoding paradigm, this may appear more natural. However, for our work, I believe we really only need to consider the case where we are prompted by a string in the prefix of the grammar. This actually implies we really only need to consider a single IntermediateLanguage: the set of strings reachable from $\epsilon$.
\begin{theorem}[Checker Intermediate Language Correct]
    \label{def:CheckerIntermediateLanguageCorrect}
    \uses{def:CheckerIntermediateLanguage,def:ConstrainedDecoding}
    Let $C$ be an checker and suppose that $x \in \langint$. Then:
    \begin{enumerate}
        \item If $w \in \langint$ and $x$ is a prefix of $w$, there exists a language model $M$ such that ConstrainedDecoding on $M$, $C$, $x$ results in $w$ being an intermediate value of $x$.
        \item Let $M$ be an arbitrary language model. If constrained decoding on $M, C, x$ encounters $w$ as an intermediate value of $x$, then $w \in \langint$.
    \end{enumerate}
\end{theorem}
We might not need this theorem as it may be awkward to define intermediate values of $x$.

\begin{definition}[Checker Language]
    \label{def:CheckerLanguage}
    \uses{def:CheckerIntermediateLanguage,def:Checker}
     Let $C$ be a checker. $w$ is a member of the Checker Language $\lang$ if and only if $w \in \langint$ and $\$ \in C(w)$. 
\end{definition}
NB: unless a checker is \emph{sound}, do note that strings in the checker intermediate language may not necessarily be a prefix of a string in the checker language (in this case, the checker never provides the end of string option). 

\begin{theorem}[Checker Language Correct]
    \label{def:CheckerLanguageCorrect}
    \uses{def:CheckerLanguage,def:ConstrainedDecoding}
    Let $C$ be an checker. Suppose that $x \in \langint$. Then: 
    \begin{enumerate}
        \item If $w \in \lang$ and $x$ is a prefix of $w$, there exists a language model $M$ such that ConstrainedDecoding on $M$, $C$, $x$ produces $w$. Informally, any word that we can hope to produce from the given prompt, is producible.
        \item Let $M$ be an arbitrary language model. If constrained decoding on $M, C, x$ terminates and returns some value $v$, then $v \in \lang$. Informally, the language captures all producible strings.
    \end{enumerate}
\end{theorem}

\begin{definition}[Checker Sound]
    \label{def:CheckerSound}
    \uses{def:Checker,def:Vocabulary}
    A checker is \emph{sound} if it is well-behaved, adhering to the two following rules:

    \begin{enumerate}
        \item $w \in \langint$ implies that there exists some $w' \in \lang$ such that $w$ is a prefix of $w'$. Informally, you never run into a case where a language model has no choices for what to output.
        \item $w \in \langint$ implies that if $w = w'v$ for some $w' \in \alphabet^*, v \in \vocab$, then $w' \in \langint$ and $v \in C(w')$. Informally, the checker is path independent.
    \end{enumerate}

    In particular, this implies that $\langint$ is exactly the set of prefixes of checker language $\lang$.
\end{definition}

\begin{definition}[Checker Complete]
    \label{def:CheckerComplete}
    \uses{def:Checker, def:CheckerIntermediateLanguage, def:CheckerLanguage}
    A checker for a language $\lang$ is \emph{complete} if the checker language and $\lang$ are equivalent, and the checker intermediate language and $\langpre$ are equivalent. 
\end{definition}


