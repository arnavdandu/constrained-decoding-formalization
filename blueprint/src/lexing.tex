\subsection{Lexing}
Results related to the lexing and detokenizing.

\begin{definition}[Lexer Regular Expression Specification]
	\label{def:LexerRESpecification}
    \uses{def:FSA, def:RE}
    A lexer regular expression specification, is given as a finite set of pairs, each consisting of a regular expression $R^i$ and a terminal symbol $T^i \in \terms$.
\end{definition}

\begin{definition}[RE to FSA]
    \label{def:BuildFSA}
    Use standard construction to a build an FSA from an input regular expression.
\end{definition}

\begin{theorem}[REEquivFSA]
    \label{thm:REEquivFSA}
    \uses{def:FSA, def:BuildFSA}
    The language of the resultant FSA from the build FSA procedure is exactly the language of the input regular expression.
\end{theorem}

\begin{definition}[Lexer Automata Specification]
	\label{def:LexerASpecification}
	\leanok
    \uses{def:FSA, def:LexerRESpecification, def:BuildFSA}
    From a lexer regular expression specification, if one applies the Build FSA procedure to each regular expression, we get a finite set of pairs, each consisting of an automaton $\automaton^i = (Q^i, \alphabet^i, q_{0}^{i}, \delta^i, F^i)$ and a terminal symbol $T^i \in \terms$.

    Hereafter, lexer specification is taken to mean lexer automata specification.
\end{definition}

\begin{definition}[Partial Lexer]
    \label{def:PartialLexer}
    \uses{def:LexerASpecification, def:Lexer}
    \leanok
    \begin{itemize}
    \item $\lexer(\epsilon) = (\epsilon, \epsilon)$
    \item Given $\lexer(w) = (T_1 \ldots T_k, w_r)$, the value of $\lexer(wc)$ is
        \begin{itemize}
        \item $(T_1 \ldots T_k T^j \$, \epsilon)$ if $c = \eos$ and $w_r \in \lang(\automaton^j)$ for some $j$
        \item $(T_1 \ldots T_k, w_r c)$ if $c \neq \eos$ and $w_r c \in \langpre(\automaton^i)$ for some $i$
        \item $(T_1 \ldots T_k T^j, c)$ if $c \neq \eos$ and $w_r \in \lang(\automaton^j)$ but $w_r c \notin \langpre(\automaton^i)$ for all $i$
        \item $\bot$ otherwise % (i.e., $w_r \notin \lang(\automaton^i)$ and $w_r c \notin \langpre(\automaton^i)$ for all $i$)
        \end{itemize}
    \end{itemize}
\end{definition}

\begin{theorem}[Lex returns iff input is Language of Lexer Specification]
    \label{thm:LexSplit}
    \uses{def:PartialLexer,def:LexerASpecification}
    Let $\lang(A_i)$ be the languages of each input automata.
    If $\lexer$ fails to return on some input string $w$, then among all possible partitions of $w$ into a set of strings $w_1 \ldots w_k$, there exists an $i$ such that $w_i \notin \lang(A_j)$ for all j. Conversely, if $\texttt{Lex} = (T_1, \ldots T_k, \epsilon)$, then $w$ may be partitioned into $w_1 \ldots w_k$ such that $w$ is the concatenation of $w_1 \ldots w_k$ and each $w_i \in \lang(A_j)$ for some $j$.
\end{theorem}
We can possibly make the second direction stronger and mention how the partition it produces is related to maximal munching / 1-lookahead. Of course, an easy corollary is that this relates to the original lexer specification in terms of the regular expressions.

\begin{definition}[DetokenizingFST]
    \label{def:DetokenizingFST}
    \uses{def:FST}
    \[
        \begin{array}{l}
            \textbf{Input:} \text{Vocabulary } \mathcal{V} \subseteq \Sigma^+ \\
            \textbf{Output:} \text{FST } \mathcal{T}_{\mathcal{V}} = (\mathcal{V}, \Sigma, Q, q_0, \delta, F) \\
            Q := \{q_\epsilon\},\quad F := \{q_\epsilon\},\quad q_0 := q_\epsilon,\quad \delta := \emptyset \\
            \text{for } c_1 \ldots c_k \in \mathcal{V} \text{ do} \\
            \quad q_{\text{prev}} := q_\epsilon \\
            \quad \text{for } i = 1 \text{ to } k-1 \text{ do} \\
            \quad\quad Q := Q \cup \{q_{c_1 \ldots c_i}\} \\
            \quad\quad \delta := \delta \cup \{ q_{\text{prev}} \xrightarrow{c_i:\epsilon} q_{c_1 \ldots c_i} \} \\
            \quad\quad q_{\text{prev}} := q_{c_1 \ldots c_i} \\
            \quad \delta := \delta \cup \{ q_{\text{prev}} \xrightarrow{c_k:c_1\ldots c_k} q_\epsilon \} \\
            \textbf{return } \mathcal{T}_{\mathcal{V}} = (\mathcal{V}, \Sigma, Q, q_0, \delta, F)
        \end{array}
    \]
\end{definition}

\begin{definition}[Lexing Transducer]
    \label{def:LexingTransducer}
    \uses{def:FSA, def:FST}
    \[
        \begin{array}{l}
            \textbf{Input:} \text{FSA } \mathcal{A} = (\Sigma, Q, q_0, \delta, F), \text{ Output alphabet } \Gamma \\
            \textbf{Output:} \text{FST } \mathcal{T}_A = (\Sigma, \Gamma, Q, q_0, \delta_{\text{FST}}, F_{\text{FST}}) \\
            \delta_{\text{FST}} := \{ q \xrightarrow{c:\epsilon} q' \mid q \xrightarrow{c} q' \in \delta \} \\
            F_{\text{FST}} := \{q_0\} \\
            \text{for state } q \text{ that recognizes language token } T \in \Gamma \text{ do} \\
            \quad \text{for } (c, q') \text{ s.t. } \exists q''. q \xrightarrow{c} q'' \notin \delta \text{ and } q_0 \xrightarrow{c} q' \in \delta \text{ do} \\
            \quad\quad \delta_{\text{FST}} := \delta_{\text{FST}} \cup \{ q \xrightarrow{c:T} q' \} \\
            \quad \delta_{\text{FST}} := \delta_{\text{FST}} \cup \{ q \xrightarrow{\text{EOS}:T\Phi} q_0 \} \\
            \textbf{return } \mathcal{T}_A = (\Sigma, \Gamma, Q, q_0, \delta_{\text{FST}}, F_{\text{FST}})
        \end{array}
    \]
\end{definition}

\begin{theorem}[Lexing Transducer Equivalent to Lex]
    \label{thm:LexingTransducerLexEquiv}
    \uses{def:PartialLexer, def:LexingTransducer}
    \begin{enumerate}
        \item If $\lexer(w)$ fails to return for some input string $w$, then the transducer does not accept $w$
        \item If $\lexer(w) = (T_1, \ldots T_k, \epsilon)$, then the transducer produces $T_1 \ldots T_k$ upon reading $w$.
    \end{enumerate}
\end{theorem}

\begin{proof}
    (incomplete)
    Let \(\mathcal{T}_A = (Q, \Sigma, \Gamma, q_0, \delta, F)\) be a lexing transducer for the lexer specification \(\{(\mathcal{A}^i, T^i)\}_i\). Then 
\[
    q_0 \xrightarrow{w : T_1 \ldots T_k}{}^* q' \in \delta^* \quad \text{if and only if} \quad \lexer(w) = (T_1 \ldots T_k, w_r)
    \]
    for some \(w_r \in \Sigma^*\) and \(q' \in Q\) such that \(q_0 \xrightarrow{w_r : \epsilon}{}^* q'\).
\end{proof}

\begin{theorem}[Lexing Transducer Memoryless]
    \label{thm:LexingTransducerMemoryless}
    Let $\transducer_{\automaton \circ \vocab} = (Q, \vocab, \terms, q_0, \delta, F)$ be a token-level lexing transducer for the lexer specification $\{(\automaton^i, T^i)\}_i$ and vocabulary $\vocab \subseteq \alphabet^+$. Then $q_0 \xrightarrow{w:T_1 \ldots T_k}^{\ast} q' \in \delta^\ast$
    if and only if $\lexer(w) = (T_1 \ldots T_k, w_r)$ for some $w_r \in \alphabet^\ast$ and $q' \in Q$ such that $q_0 \xrightarrow{w_r:\epsilon}^\ast q'$.
\end{theorem}

\begin{theorem}[Composite Transducer Equivalent to Lex]
    \label{thm:CompositeLexTransducerLexEquiv}
    \uses{thm:LexingTransducerLexEquiv,def:DetokenizingFST}
    Let $\transducer_{\automaton \circ \vocab}$ denote the (determinized) composition of the lexer and transducer. Let $F : \vocab^{*} \to \alphabet^{*}$ denote the map that transforms strings over $\vocab$ to strings over $\alphabet$. Let $v \in \vocab^{*}$.

    \begin{enumerate}
        \item If $\lexer(F(v))$ fails to return, then $\transducer_{\automaton \circ \vocab}$ does not accept $v$
        \item If $\lexer(F(v)) = (T_1, \ldots T_k, \epsilon)$, then the transducer produces $T_1 \ldots T_k$ upon reading $v$.
    \end{enumerate}
\end{theorem}
\begin{proof}
    This should largely follow from the correctness of composition (and will likely be a lemma)
\end{proof}
