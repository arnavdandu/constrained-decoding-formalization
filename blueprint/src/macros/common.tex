% In this file you should put all LaTeX macros and settings to be used both by
% the pdf version and the web version.
% This should be most of your macros.

% The theorem-like environments defined below are those that appear by default
% in the dependency graph. See the README of leanblueprint if you need help to
% customize this. 
% The configuration below use the theorem counter for all those environments
% (this is what the [theorem] arguments mean) and never resets it.
% If you want for instance to number them within chapters then you can add
% [chapter] at the end of the next line.
\newcommand{\nbc}[3]{
	{\colorbox{#3}{\bfseries\sffamily\scriptsize\textcolor{white}{#1}}}
	{\textcolor{#3}{\sf\small \textit{#2}}}
}

\newcommand{\mbc}[1]{{\color{red}[Manu: #1]}}
\newcommand\kh[1]{\nbc{KP}{#1}{brown}}
\newcommand{\khchanged}[1]{\textcolor{violet}{#1}}

\newcommand{\pfun}{\mathrel{\ooalign{\hfil$\mapstochar$\hfil\cr$\to$\cr}}}
\newcommand{\samples}{S}
\newcommand{\indicator}{\mathbbm{1}}
\newcommand{\grammar}{\mathcal{G}}
\newcommand{\nonterms}{\mathcal{N}}
\newcommand{\terms}{\Gamma}
\newcommand{\alphabet}{\Sigma}
\newcommand{\lang}{\mathcal{L}}
\newcommand{\prefix}{\mathrm{prefix}}
\newcommand{\langpre}{\lang_{\mathrm{prefix}}}
\newcommand{\langint}{\lang_{\mathrm{int}}}
\newcommand{\ruleset}{\mathcal{R}}
\newcommand{\start}{S}
\newcommand{\term}{T}
\newcommand{\nterm}{A}
\newcommand{\sent}{w}
\newcommand{\token}{t}
\newcommand{\ev}{\mathbb{E}}
\newcommand{\prob}{P}
\newcommand{\probgrammar}{Q}
\newcommand{\probgrammarpg}[2]{\probgrammar^{{#1},{#2}}}
\newcommand{\probgcd}{\tilde\probgrammar_{\mathrm{GCD}}}
\newcommand{\evgrammar}{c}
\newcommand{\tevgrammar}{\tilde c}
\newcommand{\step}{\Rightarrow}
\newcommand{\manystep}{\Rightarrow^*}
\newcommand{\eos}{\texttt{EOS}\xspace}

\newcommand{\vocab}{\mathcal{V}}
\newcommand{\xvocab}{\mathcal{V'}}

\newcommand{\cl}{\mathrm{cl}}
\newcommand{\epscl}{\cl_{\epsilon}}

\newcommand{\grammatical}{V_\grammar}
\newcommand{\nongrammar}{NG}

\newcommand{\checker}{\mathcal{C}}
\newcommand{\automaton}{\mathcal{A}}
\newcommand{\pushdown}{\mathcal{P}}
\newcommand{\transducer}{\mathcal{T}}
\newcommand{\lextrans}{\transducer_{\texttt{lex}}}
\newcommand{\lexer}{\mathtt{Lex}}

\newcommand{\inversetable}{T_{\textrm{inv}}}
\newcommand{\realizable}[2]{\mathit{Re}_{#2 \circ #1}}
\newcommand{\producible}{\mathit{Prod}}
